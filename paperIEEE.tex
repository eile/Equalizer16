
%% bare_jrnl_compsoc.tex
%% V1.4b
%% 2015/08/26
%% by Michael Shell
%% See:
%% http://www.michaelshell.org/
%% for current contact information.
%%
%% This is a skeleton file demonstrating the use of IEEEtran.cls
%% (requires IEEEtran.cls version 1.8b or later) with an IEEE
%% Computer Society journal paper.
%%
%% Support sites:
%% http://www.michaelshell.org/tex/ieeetran/
%% http://www.ctan.org/pkg/ieeetran
%% and
%% http://www.ieee.org/

%%*************************************************************************
%% Legal Notice:
%% This code is offered as-is without any warranty either expressed or
%% implied; without even the implied warranty of MERCHANTABILITY or
%% FITNESS FOR A PARTICULAR PURPOSE!
%% User assumes all risk.
%% In no event shall the IEEE or any contributor to this code be liable for
%% any damages or losses, including, but not limited to, incidental,
%% consequential, or any other damages, resulting from the use or misuse
%% of any information contained here.
%%
%% All comments are the opinions of their respective authors and are not
%% necessarily endorsed by the IEEE.
%%
%% This work is distributed under the LaTeX Project Public License (LPPL)
%% ( http://www.latex-project.org/ ) version 1.3, and may be freely used,
%% distributed and modified. A copy of the LPPL, version 1.3, is included
%% in the base LaTeX documentation of all distributions of LaTeX released
%% 2003/12/01 or later.
%% Retain all contribution notices and credits.
%% ** Modified files should be clearly indicated as such, including  **
%% ** renaming them and changing author support contact information. **
%%*************************************************************************


% *** Authors should verify (and, if needed, correct) their LaTeX system  ***
% *** with the testflow diagnostic prior to trusting their LaTeX platform ***
% *** with production work. The IEEE's font choices and paper sizes can   ***
% *** trigger bugs that do not appear when using other class files.       ***                          ***
% The testflow support page is at:
% http://www.michaelshell.org/tex/testflow/


\documentclass[10pt,journal,compsoc]{IEEEtran}
%
% If IEEEtran.cls has not been installed into the LaTeX system files,
% manually specify the path to it like:
% \documentclass[10pt,journal,compsoc]{../sty/IEEEtran}





% Some very useful LaTeX packages include:
% (uncomment the ones you want to load)


% *** MISC UTILITY PACKAGES ***
%
%\usepackage{ifpdf}
% Heiko Oberdiek's ifpdf.sty is very useful if you need conditional
% compilation based on whether the output is pdf or dvi.
% usage:
% \ifpdf
%   % pdf code
% \else
%   % dvi code
% \fi
% The latest version of ifpdf.sty can be obtained from:
% http://www.ctan.org/pkg/ifpdf
% Also, note that IEEEtran.cls V1.7 and later provides a builtin
% \ifCLASSINFOpdf conditional that works the same way.
% When switching from latex to pdflatex and vice-versa, the compiler may
% have to be run twice to clear warning/error messages.






% *** CITATION PACKAGES ***
%
\ifCLASSOPTIONcompsoc
  % IEEE Computer Society needs nocompress option
  % requires cite.sty v4.0 or later (November 2003)
  \usepackage[nocompress]{cite}
\else
  % normal IEEE
  \usepackage{cite}
\fi
% cite.sty was written by Donald Arseneau
% V1.6 and later of IEEEtran pre-defines the format of the cite.sty package
% \cite{} output to follow that of the IEEE. Loading the cite package will
% result in citation numbers being automatically sorted and properly
% "compressed/ranged". e.g., [1], [9], [2], [7], [5], [6] without using
% cite.sty will become [1], [2], [5]--[7], [9] using cite.sty. cite.sty's
% \cite will automatically add leading space, if needed. Use cite.sty's
% noadjust option (cite.sty V3.8 and later) if you want to turn this off
% such as if a citation ever needs to be enclosed in parenthesis.
% cite.sty is already installed on most LaTeX systems. Be sure and use
% version 5.0 (2009-03-20) and later if using hyperref.sty.
% The latest version can be obtained at:
% http://www.ctan.org/pkg/cite
% The documentation is contained in the cite.sty file itself.
%
% Note that some packages require special options to format as the Computer
% Society requires. In particular, Computer Society  papers do not use
% compressed citation ranges as is done in typical IEEE papers
% (e.g., [1]-[4]). Instead, they list every citation separately in order
% (e.g., [1], [2], [3], [4]). To get the latter we need to load the cite
% package with the nocompress option which is supported by cite.sty v4.0
% and later. Note also the use of a CLASSOPTION conditional provided by
% IEEEtran.cls V1.7 and later.





% *** GRAPHICS RELATED PACKAGES ***
%
\ifCLASSINFOpdf
  % \usepackage[pdftex]{graphicx}
  % declare the path(s) where your graphic files are
  % \graphicspath{{../pdf/}{../jpeg/}}
  % and their extensions so you won't have to specify these with
  % every instance of \includegraphics
  % \DeclareGraphicsExtensions{.pdf,.jpeg,.png}
\else
  % or other class option (dvipsone, dvipdf, if not using dvips). graphicx
  % will default to the driver specified in the system graphics.cfg if no
  % driver is specified.
  % \usepackage[dvips]{graphicx}
  % declare the path(s) where your graphic files are
  % \graphicspath{{../eps/}}
  % and their extensions so you won't have to specify these with
  % every instance of \includegraphics
  % \DeclareGraphicsExtensions{.eps}
\fi
% graphicx was written by David Carlisle and Sebastian Rahtz. It is
% required if you want graphics, photos, etc. graphicx.sty is already
% installed on most LaTeX systems. The latest version and documentation
% can be obtained at:
% http://www.ctan.org/pkg/graphicx
% Another good source of documentation is "Using Imported Graphics in
% LaTeX2e" by Keith Reckdahl which can be found at:
% http://www.ctan.org/pkg/epslatex
%
% latex, and pdflatex in dvi mode, support graphics in encapsulated
% postscript (.eps) format. pdflatex in pdf mode supports graphics
% in .pdf, .jpeg, .png and .mps (metapost) formats. Users should ensure
% that all non-photo figures use a vector format (.eps, .pdf, .mps) and
% not a bitmapped formats (.jpeg, .png). The IEEE frowns on bitmapped formats
% which can result in "jaggedy"/blurry rendering of lines and letters as
% well as large increases in file sizes.
%
% You can find documentation about the pdfTeX application at:
% http://www.tug.org/applications/pdftex






% *** MATH PACKAGES ***
%
%\usepackage{amsmath}
% A popular package from the American Mathematical Society that provides
% many useful and powerful commands for dealing with mathematics.
%
% Note that the amsmath package sets \interdisplaylinepenalty to 10000
% thus preventing page breaks from occurring within multiline equations. Use:
%\interdisplaylinepenalty=2500
% after loading amsmath to restore such page breaks as IEEEtran.cls normally
% does. amsmath.sty is already installed on most LaTeX systems. The latest
% version and documentation can be obtained at:
% http://www.ctan.org/pkg/amsmath





% *** SPECIALIZED LIST PACKAGES ***
%
%\usepackage{algorithmic}
% algorithmic.sty was written by Peter Williams and Rogerio Brito.
% This package provides an algorithmic environment fo describing algorithms.
% You can use the algorithmic environment in-text or within a figure
% environment to provide for a floating algorithm. Do NOT use the algorithm
% floating environment provided by algorithm.sty (by the same authors) or
% algorithm2e.sty (by Christophe Fiorio) as the IEEE does not use dedicated
% algorithm float types and packages that provide these will not provide
% correct IEEE style captions. The latest version and documentation of
% algorithmic.sty can be obtained at:
% http://www.ctan.org/pkg/algorithms
% Also of interest may be the (relatively newer and more customizable)
% algorithmicx.sty package by Szasz Janos:
% http://www.ctan.org/pkg/algorithmicx




% *** ALIGNMENT PACKAGES ***
%
%\usepackage{array}
% Frank Mittelbach's and David Carlisle's array.sty patches and improves
% the standard LaTeX2e array and tabular environments to provide better
% appearance and additional user controls. As the default LaTeX2e table
% generation code is lacking to the point of almost being broken with
% respect to the quality of the end results, all users are strongly
% advised to use an enhanced (at the very least that provided by array.sty)
% set of table tools. array.sty is already installed on most systems. The
% latest version and documentation can be obtained at:
% http://www.ctan.org/pkg/array


% IEEEtran contains the IEEEeqnarray family of commands that can be used to
% generate multiline equations as well as matrices, tables, etc., of high
% quality.




% *** SUBFIGURE PACKAGES ***
%\ifCLASSOPTIONcompsoc
%  \usepackage[caption=false,font=footnotesize,labelfont=sf,textfont=sf]{subfig}
%\else
%  \usepackage[caption=false,font=footnotesize]{subfig}
%\fi
% subfig.sty, written by Steven Douglas Cochran, is the modern replacement
% for subfigure.sty, the latter of which is no longer maintained and is
% incompatible with some LaTeX packages including fixltx2e. However,
% subfig.sty requires and automatically loads Axel Sommerfeldt's caption.sty
% which will override IEEEtran.cls' handling of captions and this will result
% in non-IEEE style figure/table captions. To prevent this problem, be sure
% and invoke subfig.sty's "caption=false" package option (available since
% subfig.sty version 1.3, 2005/06/28) as this is will preserve IEEEtran.cls
% handling of captions.
% Note that the Computer Society format requires a sans serif font rather
% than the serif font used in traditional IEEE formatting and thus the need
% to invoke different subfig.sty package options depending on whether
% compsoc mode has been enabled.
%
% The latest version and documentation of subfig.sty can be obtained at:
% http://www.ctan.org/pkg/subfig




% *** FLOAT PACKAGES ***
%
%\usepackage{fixltx2e}
% fixltx2e, the successor to the earlier fix2col.sty, was written by
% Frank Mittelbach and David Carlisle. This package corrects a few problems
% in the LaTeX2e kernel, the most notable of which is that in current
% LaTeX2e releases, the ordering of single and double column floats is not
% guaranteed to be preserved. Thus, an unpatched LaTeX2e can allow a
% single column figure to be placed prior to an earlier double column
% figure.
% Be aware that LaTeX2e kernels dated 2015 and later have fixltx2e.sty's
% corrections already built into the system in which case a warning will
% be issued if an attempt is made to load fixltx2e.sty as it is no longer
% needed.
% The latest version and documentation can be found at:
% http://www.ctan.org/pkg/fixltx2e


%\usepackage{stfloats}
% stfloats.sty was written by Sigitas Tolusis. This package gives LaTeX2e
% the ability to do double column floats at the bottom of the page as well
% as the top. (e.g., "\begin{figure*}[!b]" is not normally possible in
% LaTeX2e). It also provides a command:
%\fnbelowfloat
% to enable the placement of footnotes below bottom floats (the standard
% LaTeX2e kernel puts them above bottom floats). This is an invasive package
% which rewrites many portions of the LaTeX2e float routines. It may not work
% with other packages that modify the LaTeX2e float routines. The latest
% version and documentation can be obtained at:
% http://www.ctan.org/pkg/stfloats
% Do not use the stfloats baselinefloat ability as the IEEE does not allow
% \baselineskip to stretch. Authors submitting work to the IEEE should note
% that the IEEE rarely uses double column equations and that authors should try
% to avoid such use. Do not be tempted to use the cuted.sty or midfloat.sty
% packages (also by Sigitas Tolusis) as the IEEE does not format its papers in
% such ways.
% Do not attempt to use stfloats with fixltx2e as they are incompatible.
% Instead, use Morten Hogholm'a dblfloatfix which combines the features
% of both fixltx2e and stfloats:
%
% \usepackage{dblfloatfix}
% The latest version can be found at:
% http://www.ctan.org/pkg/dblfloatfix




%\ifCLASSOPTIONcaptionsoff
%  \usepackage[nomarkers]{endfloat}
% \let\MYoriglatexcaption\caption
% \renewcommand{\caption}[2][\relax]{\MYoriglatexcaption[#2]{#2}}
%\fi
% endfloat.sty was written by James Darrell McCauley, Jeff Goldberg and
% Axel Sommerfeldt. This package may be useful when used in conjunction with
% IEEEtran.cls'  captionsoff option. Some IEEE journals/societies require that
% submissions have lists of figures/tables at the end of the paper and that
% figures/tables without any captions are placed on a page by themselves at
% the end of the document. If needed, the draftcls IEEEtran class option or
% \CLASSINPUTbaselinestretch interface can be used to increase the line
% spacing as well. Be sure and use the nomarkers option of endfloat to
% prevent endfloat from "marking" where the figures would have been placed
% in the text. The two hack lines of code above are a slight modification of
% that suggested by in the endfloat docs (section 8.4.1) to ensure that
% the full captions always appear in the list of figures/tables - even if
% the user used the short optional argument of \caption[]{}.
% IEEE papers do not typically make use of \caption[]'s optional argument,
% so this should not be an issue. A similar trick can be used to disable
% captions of packages such as subfig.sty that lack options to turn off
% the subcaptions:
% For subfig.sty:
% \let\MYorigsubfloat\subfloat
% \renewcommand{\subfloat}[2][\relax]{\MYorigsubfloat[]{#2}}
% However, the above trick will not work if both optional arguments of
% the \subfloat command are used. Furthermore, there needs to be a
% description of each subfigure *somewhere* and endfloat does not add
% subfigure captions to its list of figures. Thus, the best approach is to
% avoid the use of subfigure captions (many IEEE journals avoid them anyway)
% and instead reference/explain all the subfigures within the main caption.
% The latest version of endfloat.sty and its documentation can obtained at:
% http://www.ctan.org/pkg/endfloat
%
% The IEEEtran \ifCLASSOPTIONcaptionsoff conditional can also be used
% later in the document, say, to conditionally put the References on a
% page by themselves.




% *** PDF, URL AND HYPERLINK PACKAGES ***
%
%\usepackage{url}
% url.sty was written by Donald Arseneau. It provides better support for
% handling and breaking URLs. url.sty is already installed on most LaTeX
% systems. The latest version and documentation can be obtained at:
% http://www.ctan.org/pkg/url
% Basically, \url{my_url_here}.





% *** Do not adjust lengths that control margins, column widths, etc. ***
% *** Do not use packages that alter fonts (such as pslatex).         ***
% There should be no need to do such things with IEEEtran.cls V1.6 and later.
% (Unless specifically asked to do so by the journal or conference you plan
% to submit to, of course. )

\usepackage{mathptmx}
\usepackage{graphicx}
\usepackage{times}
%\usepackage{tabularx}
\usepackage{amsmath}
%\usepackage{url}
\usepackage{subfigure}
\usepackage{multirow}
\usepackage{paralist}

\usepackage{color}
\definecolor{RED}{rgb}{1,0,0}
\definecolor{BLUE}{rgb}{0,0,1}
\newcommand{\FIXME}[1]{\textbf{\color{BLUE}{FIXME: #1}}}
\newcommand{\sref}[1]{Section~\ref{#1}}
\newcommand{\fig}[1]{Figure~\ref{#1}}

% suppress  single floating lines on top (widow) and bottom (club)
%  10000 is infinity
%  tradeoff: possible underful vboxes
\clubpenalty=10000
\widowpenalty=10000

% correct bad hyphenation here
\hyphenation{op-tical net-works semi-conduc-tor}


\begin{document}
%
% paper title
% Titles are generally capitalized except for words such as a, an, and, as,
% at, but, by, for, in, nor, of, on, or, the, to and up, which are usually
% not capitalized unless they are the first or last word of the title.
% Linebreaks \\ can be used within to get better formatting as desired.
% Do not put math or special symbols in the title.
\title{Equalizer: A Mature Parallel Rendering Framework}
%
%
% author names and IEEE memberships
% note positions of commas and nonbreaking spaces ( ~ ) LaTeX will not break
% a structure at a ~ so this keeps an author's name from being broken across
% two lines.
% use \thanks{} to gain access to the first footnote area
% a separate \thanks must be used for each paragraph as LaTeX2e's \thanks
% was not built to handle multiple paragraphs
%
%
%\IEEEcompsocitemizethanks is a special \thanks that produces the bulleted
% lists the Computer Society journals use for "first footnote" author
% affiliations. Use \IEEEcompsocthanksitem which works much like \item
% for each affiliation group. When not in compsoc mode,
% \IEEEcompsocitemizethanks becomes like \thanks and
% \IEEEcompsocthanksitem becomes a line break with idention. This
% facilitates dual compilation, although admittedly the differences in the
% desired content of \author between the different types of papers makes a
% one-size-fits-all approach a daunting prospect. For instance, compsoc
% journal papers have the author affiliations above the "Manuscript
% received ..."  text while in non-compsoc journals this is reversed. Sigh.

%% Author and Affiliation (multiple authors with multiple affiliations)
\author{Stefan Eilemann\thanks{email: eilemann@gmail.com} \\ %
\and Renato Pajarola\thanks{email: pajarola@acm.org}}


\author{Stefan~Eilemann, David~Steiner and Renato Pajarola% <-this % stops a space
\IEEEcompsocitemizethanks{\IEEEcompsocthanksitem All authors are with the Visualization and MultiMedia Lab, Department of Informatics, University of Z\"urich.
\IEEEcompsocthanksitem S. Eilemann is also with the Blue Brain Project, Ecole
Polytechnique Federale de Lausanne.}}

% note the % following the last \IEEEmembership and also \thanks -
% these prevent an unwanted space from occurring between the last author name
% and the end of the author line. i.e., if you had this:
%
% \author{....lastname \thanks{...} \thanks{...} }
%                     ^------------^------------^----Do not want these spaces!
%
% a space would be appended to the last name and could cause every name on that
% line to be shifted left slightly. This is one of those "LaTeX things". For
% instance, "\textbf{A} \textbf{B}" will typeset as "A B" not "AB". To get
% "AB" then you have to do: "\textbf{A}\textbf{B}"
% \thanks is no different in this regard, so shield the last } of each \thanks
% that ends a line with a % and do not let a space in before the next \thanks.
% Spaces after \IEEEmembership other than the last one are OK (and needed) as
% you are supposed to have spaces between the names. For what it is worth,
% this is a minor point as most people would not even notice if the said evil
% space somehow managed to creep in.


% The paper headers
\markboth{Journal of \LaTeX\ Class Files,~Vol.~14, No.~8, August~2015}%
{Shell \MakeLowercase{\textit{et al.}}: Bare Demo of IEEEtran.cls for Computer Society Journals}
% The only time the second header will appear is for the odd numbered pages
% after the title page when using the twoside option.
%
% *** Note that you probably will NOT want to include the author's ***
% *** name in the headers of peer review papers.                   ***
% You can use \ifCLASSOPTIONpeerreview for conditional compilation here if
% you desire.



% The publisher's ID mark at the bottom of the page is less important with
% Computer Society journal papers as those publications place the marks
% outside of the main text columns and, therefore, unlike regular IEEE
% journals, the available text space is not reduced by their presence.
% If you want to put a publisher's ID mark on the page you can do it like
% this:
%\IEEEpubid{0000--0000/00\$00.00~\copyright~2015 IEEE}
% or like this to get the Computer Society new two part style.
%\IEEEpubid{\makebox[\columnwidth]{\hfill 0000--0000/00/\$00.00~\copyright~2015 IEEE}%
%\hspace{\columnsep}\makebox[\columnwidth]{Published by the IEEE Computer Society\hfill}}
% Remember, if you use this you must call \IEEEpubidadjcol in the second
% column for its text to clear the IEEEpubid mark (Computer Society jorunal
% papers don't need this extra clearance.)



% use for special paper notices
%\IEEEspecialpapernotice{(Invited Paper)}

% for Computer Society papers, we must declare the abstract and index terms
% PRIOR to the title within the \IEEEtitleabstractindextext IEEEtran
% command as these need to go into the title area created by \maketitle.
% As a general rule, do not put math, special symbols or citations
% in the abstract or keywords.
\IEEEtitleabstractindextext{%
\begin{abstract}
  We present the features, algorithms and system integration necessary to
  implement a parallel rendering framework usable in a wide range of real-world
  research and industry applications, based on the basic architecture and
  implementation of the Equalizer parallel rendering framework presented in
  \cite{EMP:09}.
\end{abstract}

% Note that keywords are not normally used for peerreview papers.
\begin{IEEEkeywords}
Parallel Rendering, Scalable Visualization, Cluster Graphics, Immersive Environments, Display Walls
\end{IEEEkeywords}}


% make the title area
\maketitle


% To allow for easy dual compilation without having to reenter the
% abstract/keywords data, the \IEEEtitleabstractindextext text will
% not be used in maketitle, but will appear (i.e., to be "transported")
% here as \IEEEdisplaynontitleabstractindextext when the compsoc
% or transmag modes are not selected <OR> if conference mode is selected
% - because all conference papers position the abstract like regular
% papers do.
\IEEEdisplaynontitleabstractindextext
% \IEEEdisplaynontitleabstractindextext has no effect when using
% compsoc or transmag under a non-conference mode.



% For peer review papers, you can put extra information on the cover
% page as needed:
% \ifCLASSOPTIONpeerreview
% \begin{center} \bfseries EDICS Category: 3-BBND \end{center}
% \fi
%
% For peerreview papers, this IEEEtran command inserts a page break and
% creates the second title. It will be ignored for other modes.
\IEEEpeerreviewmaketitle

%----------------------------------------------------------------------
\begin{figure*}[ht]\center
  \includegraphics[width=2\columnwidth]{images/teaser} \\
  (a) \hfil \hfil (b) \hfil \hfil (c)
  \vspace{-2mm}
  \caption{Example Equalizer application use cases: (a) 192 Megapixel CAVE at
    KAUST running RTNeuron, (b) Immersive HMD with external tracked and
    untracked views running RTT DeltaGen for virtual car usability studies (c)
    Cave2 running a molecular visualization build using Omegalib.}
  \label{FIG_teaser}
\end{figure*}

%----------------------------------------------------------------------
% Computer Society journal (but not conference!) papers do something unusual
% with the very first section heading (almost always called "Introduction").
% They place it ABOVE the main text! IEEEtran.cls does not automatically do
% this for you, but you can achieve this effect with the provided
% \IEEEraisesectionheading{} command. Note the need to keep any \label that
% is to refer to the section immediately after \section in the above as
% \IEEEraisesectionheading puts \section within a raised box.
\IEEEraisesectionheading{\section{Introduction}\label{sec:introduction}}
%----------------------------------------------------------------------

The continuing improvements in hardware integration lead to ever faster CPUs and GPUs, as well as higher resolution sensor and display devices. Moreover, increased hardware parallelism is applied in form of multi-core CPU workstations, massive parallel super computers, or cluster systems. Hand in hand goes the rapid growth in complexity of data sets from numerical simulations, high-resolution 3D scanning systems or bio-medical imaging, which causes interactive exploration and visualization of such large data sets to become a serious challenge. It is thus crucial for a visualization solution to take advantage of hardware accelerated scalable parallel rendering. In this systems paper we describe a new scalable parallel rendering framework called {\em Equalizer} that is aimed primarily at cluster-parallel rendering, but works as well in a shared-memory system. Cluster systems are the main focus  because workstation graphics hardware is developing faster than high-end graphics (super-) computers can absorb new developments, and also because clusters offer a better cost-performance balance.

Previous parallel rendering approaches typically failed in one of the following system requirements:
%
\begin{compactenum}\renewcommand{\labelenumi}{\alph{enumi})}
\item generic application support, instead of special domain solution
\item scalable abstraction of the graphics layer
\item exploit existing code infrastructure, such as proprietary scene graphs, molecular data structures, level-of-detail and geometry databases
\end{compactenum}

To date, generic and scalable parallel rendering frameworks  that can be adopted to a wide range of scientific visualization domains are not yet readily available. Furthermore, flexible
configurability to arbitrary cluster and display-wall configurations has also not been addressed in the past, but is of immense practical importance to scientists depending high-performance interactive visualization as a scientific tool. In this paper we present Equalizer, which is a novel flexible framework for parallel rendering that supports scalable performance, configuration flexibility, is {\em minimally invasive} with respect to adapting existing visualization applications, and is applicable to virtually any scientific visualization application domain.

The main contributions that Equalizer introduces in a single parallel rendering system, and which are presented in this paper are:
%
\begin{compactenum}\renewcommand{\labelenumi}{\roman{enumi})}
\item novel concept of compound trees for flexible configuration of graphics system resources,
\item easy specification of parallel task decomposition and image compositing choice through compound tree layouts,
\item automatic decomposition and distributed execution of rendering tasks according to compound tree,
\item support for parallel surface as well as transparent (volume) rendering through $z$-visibility as well as $\alpha$-blending compositing,
\item fully decentralized architecture providing network swap barrier (synchronization) and distributed objects functionality,
\item support for low-latency distributed frame synchronization and image compositing,
\item minimally invasive programming model.
\end{compactenum}

Equalizer is open source, available under the LGPL license from http://www.equalizergraphcis.com/, which allows it to be used both for open source and commercial applications. It is source-code portable, and has been tested on Linux, Microsoft Windows, and Mac OS X in 32 and 64 bit mode using both little endian and big endian processors.


%----------------------------------------------------------------------
\section{Related Work}
%----------------------------------------------------------------------
\label{SEC_related}

In \cite{EMP:09}, we presented the Equalizer parallel rendering framework. This
paper also summarized the work in parallel rendering up to 2009. In the
following we will present the related work published since then.

\cite{EEP:11} cross-segment load balancing

\cite{DK:11}: Framework to develop apps for multitile, thus targeted
specifically for tiled-wall visualization systems: grid configuration,
scalability (?), multiple work sessions, multiuser events. Master-slave approach
like Eq. with grid nodes running instances of the application. Is maybe more
like a GLUT and window replacement, or enhancement to deal with the OpenGL
contexts and events

\cite{CKP:12} tiled display wall virtual environment

\cite{TBD} DisplayCluster

\cite{NHM:11} ClusterGL?

Omegalib

%------------------------------------------------------------------------------
\section{Usability}
%------------------------------------------------------------------------------

In this section we present features motivated by real-world application use
cases, i.e., new functionalities rather then performance improvements. We
motivate the use case, explain the architecture and integration into our
parallel rendering framework, and, where applicable, show the steps needed to
use this functionality in applications.

\subsection{Physical and Logical Visualization Setup}

Real-world visualization setups are often complex, and having an abstract
representation of the display system can simplify the configuration
process. Real-world applications often have the need to be aware of spatial
relationship of the display setup, for example to render 2D overlays or to
configure multiple views on a tiled display wall.

We addressed this need through a new configuration section interspersed between
the node/pipe/window/channel hardware resources and the compound trees
configurating the resource usage for parallel rendering.

A typical installation consists of one projection canvas, which is one
aggregated projection surface, e.g., a tiled display wall or a CAVE. Desktop
windows are considered a canvas. Each canvas is made of one or more segments,
which are the individual outputs connected to a display or projector. Segments
can be planar or non-planar to each other, and can overlap or have gaps between
each other. A segment is referencing a channel, which defines the output area of
this segment, e.g., on a DVI connector connected to a projector.

A canvas can define a frustum, which will create default, planar sub-frusta for
all of its segments. A segment can also define a frustum, which overrides the
canvas frustum, e.g., for non-planar setups such as CAVEs or curved
screens. These frusta describe a physically-correct display setup for a Virtual
Reality installation. A canvas may have a software or hardware swap barrier,
which will synchronize the rendering of all contributing GPUs.

On each canvas, the application can display one or more views. A view is a view
on a model, in the sense used by the MVC pattern. The view class is used by
Equalizer applications to define view-specific data for rendering, e.g., a
scene, viewing mode or camera. The application process manages this data, and
the render clients receive it for rendering.

A layout groups one or more views which logically belong together. A layout is
applied to a canvas. The layout assignment can be changed at run-time by the
application. The intersection between views and segments defines which output
channels are available, and which frustum they should use for rendering. These
output channels are then used as destination channels in a compound. They are
automatically created during configuration.

A view may have a frustum description. The view's frustum overrides frusta
specified at the canvas or segment level. This is used for non-physically
correct rendering, e.g., to compare two models side-by-side on a tiled display
wall. If the view does not specify a frustum, the corresponding destination
channels will use the physically correct sub-frustum resulting from the
view/segment intersection.

\label{SEC_observer}
An observer looks at one or more views. It is described by the observer position
in the world and its eye separation. Each observer will have its own stereo
mode, focus distance and frame loop (framerate). This allows to have untracked
views and multiple tracked views, e.g., two HMDs, in the same application.

\subsection{Automatic Configuration}

Automatic configuration implements the discovery of local and remote resources
as well as the creation of typical configurations using the discovered resources
at application launch time.

The discovery is implemented in a separate library, hwsd (HardWare Service
Discovery), which uses a plugin-based approach to discover GPUs for GLX, AGL or
WGL windowing systems, as well as network interfaces on Linux, Mac OS X and
Windows. Furthermore, it detects the presence of VirtualGL to allow optimal
configuration of remote visualization clusters. The resources can be discovered
on the local workstation, and through the help of a simple daemon using the
zeroconf protocol, on a set of remote nodes within a visualization cluster. A
session identifier may be used to support multiple users on a single cluster.

The Equalizer server uses the hwsd library to discover local and remote
resources when an hwsd session name instead of a \textsf{.eqc} configuration
file is provided. A set of standard decomposition modes is configured, which can
be selected through activating the corresponding layout.

This versatile mechanism allows non-experts to configure asnd profit from
multi-GPU workstations and visualization clusters, as well as to provide system
administrators with the tools to implement easy to use integration with cluster
schedulers. This feature is transparent to Equalizer application developers.

\subsection{Qt Windowing}

Qt is a popular window system with many application developers. Unfortunately,
it imposes a different threading model for window creation and event handling
then Equalizer. In Equalizer, each GPU rendering thread is independently
responsible for creating its windows, receiving the events and eventually
dispatching them to the application process main thread. This design is
motivated by the natural threading model of X11 and WGL, and allows simple
sequential semantics between OpenGL rendering and event handling. In contrast,
Qt requires all windows and QOpenGLContext to be created from the Qt main
thread. An existing Qt window or context may be moved to a different thread, and
events are signalled from the main thread.




Challenges in threading model, architecture

\subsection{Tide Integration}

Tiled interactive display environment, parallel pixel streaming, events

\subsection{Sequel}

application, renderer, view data

%------------------------------------------------------------------------------
\section{The Collage Network Library}
%------------------------------------------------------------------------------

\subsection{Distributed, Versioned Objects}

types (instance, delta, static), versioning, multicast, compression,
serializable with dirty bits, mapping, blocking commits

\subsection{Reliable Stream Protocol}

UDP-based reliability protocol

\subsection{Infiniband RDMA}

reverse-engr impl

%------------------------------------------------------------------------------
\section{Virtual Reality}
%------------------------------------------------------------------------------

Virtual Reality is an important field for parallel rendering. It does however
require special attention to support it as a first-class citizen in a generic
parallel rendering framework. Equalizer has been used in many virtual reality
installations, such as the Cave2 (\cite{FNTTL:13}), the high-resolution C6 CAVE
at the KAUST visualization laboratory and head-mounted displays
(\fig{FIG_teaser}). In the following we lay out the features needed support
these installations. All features presented were motivated by application use
cases and have been validated with their respective users.

\subsection{Head Tracking}

Head tracking is the minimal feature needed to support immersive
installations. Equalizer does support multiple, independent tracked views
through the observer abstraction introduced in \sref{SEC_observer}. Built-in
VRPN support enables the direct, application-transparent configuration of a VRPN
tracker device. Alternatively, an application can provide a $4\times 4$ tracking
matrix defining the transformation from the canvas coordinate system to the
observer. Both CAVE-like tracking with fixed projection surfaces and HMD
tracking modes are implemented.

\subsection{Dynamic Focus Distance}

To our knowledge, all parallel rendering systems have the focal plane coincide
with the physical display surface. For better viewing comfort, we introduce a
new dynamic focus mode, where the application defines the distance of the focal
plane from the observer, based on the current view direction af the
user. Initial experiments show that this is particularly effective for objects
placed within the immersive space, that is, in front of all display segments.

\subsection{Asymmetric Eye Position}

Traditional head tracking computes the left and right eye positions by using a
configurable interocular distance and the tracking matrix. However, human heads
are not symmetric, and by measuring individual users a more precise frustum can
be computed. Equalizer supports this through the optional configuration (file or
programmatically) of individual 3D positions for the left and right eye.

\subsection{Application-specific Scaling}

Gullivers world

\subsection{Runtime Stereo Switch}

\subsection{Swap Synchronization and GPU affinity}

%------------------------------------------------------------------------------
\section{Performance}
%------------------------------------------------------------------------------

\subsection{New Decomposition Modes}

The initial version of Equalizer implemented sort-first (2D), sort-last (DB) and
stereo (EYE) decomposition. In the following we present new decomposition modes
and motivate their use case.

\subsubsection{Time-Multiplex}

Time-multiplexing, or DPlex, was already implemented in \cite{BRE:05}. While it
increases the framerate linearly, it does not decrease the latency between user
input and the corresponding output. Consequently, this decomposition mode is
mostly useful for non-interactive movie generation. It is transparent to
Equalizer applications, but does require the configuration latency to be equal
or greater than the number of source channels. Furthermore, to work in a
multi-threaded, multi-GPU configuration, the application needs to support
running the rendering threads asynchronously, as outlined in
\sref{SEC_threading}. The output frame rate of the destination channel is
smoothened using a frame rate equalizer (\sref{SEC_framerateEq}).

\subsubsection{Tiles and Chunks}

Tile and chunk decompositions are a variant of sort-first and sort-last
rendering, respectively. They decompose the scene into a predefined set of
fixed-size image tiles or database ranges. These tasks are queued and processed
by all source channels by polling a server-central queue. Prefetching ensures
that the task communication overlaps with rendering. As shown in \cite{SPEP:16},
these modes provide better performance due to being inherently load-balanced, as
long as there is an insignificant overhead for the render task setup. This mode
is transparent to Equalizer applications.

\subsubsection{Pixel}

Pixel compounds decompose the destination channel by interleaving rows or
columns in image space. They are a variant of sort-first decomposition which
works well for fill-limited applications which are not geometry bound. Source
channels cannot reduce geometry load through view frustum culling, since each
source channel has almost the same frustum (only shifted by some pixels),
but applied to a reduced 2D viewport. However, the fragment load on all source
channels is very similar due to the interleaved distribution of pixels. This
functionality is transparent to Equalizer applications, and the default
compositing implementation uses the OpenGL stencil buffer to blit pixels onto
the destination channel.

\subsubsection{Subpixel}

Subpixel compounds are similar to pixel compounds, but they decompose the work
for a single pixel, for example when using multisampling or depth of
field. Composition typically uses accumulation and averaging of all computed
fragments for a pixel. This feature is not fully transparent to the application,
since it needs to adapt (jitter or tilt) the frustum based on the iteration
executed. Furthermore, subpixel compounds interact with idle image refinements,
e.g., they can accelerate idle anti-aliasing of a scene when the camera and
scene are not changed.

\subsection{Equalizers}

Equalizer are an addition to compound trees. They modify parameters of their
respective subtree at runtime to optimize one aspect of the decomposition. Due
to their nature, they are transparent to application developers, but might have
application-accessible parameters to tune their behaviour.

\subsubsection{Sort-First and Sort-Last Load Equalizer}

Sort-first and sort-last load balancing is the most obvious optimization for
these parallel rendering modes. Our load equalizer is fully transparent for
application developers, that is, it uses a reactive approach based on past
rendering times. This assumes a reasonable frame-to-frame coherency. Our
implementation stores a 2D or 1D grid of the load, mapping the load of each
channel. The load is stored in normalized 2D/1D coordinates using
$\frac{time}{area}$ as the load, the contributing source channels are organized
in a binary tree, and then the algorithm balances the two branches of each level
by equalizing the integral over the area on each side.

We have implemented various tunable parameters allowing application developers
to optimize the load balancing based on the characteristics of their rendering
algorithm:
\begin{compactdesc}
\item[Damping] reduces frame-to-frame oscillations. It is a normalized scalar
  defining how much of the computed delta from the previous position is
  applied. The equal load distribution within the region of interest assumed by
  the load equalizer is in reality not equal, causing the load balancing to
  overshoot.
\item[Resistance] eliminates small deltas in the load balancing step. This might
  help the application to cache some computations since the frustum does not
  change each frame.
\item[Boundaries] define the modulo factor in pixels onto which a load split may
  fall. Some rendering algorithms produce artefacts related to the OpenGL raster
  position, e.g., screen door transparency, which can be eliminated by aligning
  the boundary to the pixel repetition. Furthermore, some rendering algorithms
  are sensitive to cache alignments, which can again be exploited by chosing the
  corresponding boundary.
\end{compactdesc}

\subsubsection{Cross-Segment Load Balancing}

Cross-segment load balancing addresses the optimal resource allocation of $n$
rendering resources to $m$ output channels (with $n\geq m$). The view equalizer
works in conjunction with load equalizer balancing the individual output
channels. It monitors the usage of shared source channels (across outputs) and
activates them to balance the rendering time of all outputs. In \cite{EEP:11},
we provide a detailed description and evaluation of our algorithm.

\subsubsection{Dynamic Frame Resolution}

The DFR equalizer provides a functionality similar to dynamic video resizing
\cite{MBDM:97}, that is, it maintains a constant framerate by adapting the
rendering resolution of a fill-limited application. In Equalizer, this works by
rendering into a source channel (typically on a FBO) separate to the destination
channel, and then scaling the rendering during the transfer (typically through
an on-GPU texture) to the destination channel. The DFR equalizer monitors the
rendering performance and accordingly adapts the resolution of the source
channel and zoom factor for the source to destination transfer. If the
performance and source channel resolutions allows, this will not only subsample,
but also supersample the destination channel to reduce aliasing artefacts.

\subsubsection{Frame Rate Equalizer}\label{SEC_framerateEq}

The framerate equalizer smoothens the output frame rate of a destination
channel by instructing the corresponding window to delay its buffer swap to a
minimum time between swaps. This is regularly used for time-multiplexed
decompostions, where source channels tend to drift and finish their rendering
not evenly distributed over time. This equalizer is however fully independent of
DPlex compounds, and may be used to smoothen irregular application rendering
algorithms.

\subsubsection{Monitoring}

Control workstation in VR setups

\subsection{Optimizations}

\subsubsection{Region of Interest}

The region of interest is the screen-space 2D bounding box enclosing the
geometry rendered by a single resource. We have extended the core parallel
rendering framework to use an application-provided ROI to optimize the load
equalizer as well as image compositing performance. The load equalizer uses the
ROI to refine its load grid to the regions containing data. The compositing code
uses the ROI to minimize image readback and network transmission. In
\cite{MEP:10} and \cite{EBAHMP:12}, we provide the details of the algorithm, and
show that using ROI can quadruple the rendering performance, in particular for
the costly compositing step in sort-last rendering.

\subsubsection{Asynchronous Compositing}

Asynchronous compositing pipelines rendering with compositing operations, by
executing the image readback, network transfer and image assembly from threads
running in parallel to the rendering threads. In \cite{EBAHMP:12}, we provide
the details of the implementation and experimental data showing an improvement
of the rendering performance of over 25\% for large node counts.

\subsubsection{Download and Compression Plugins}

GPU-CPU transfer plugins with optional compression (YUV, RLE) linked to CPU
compression for network transfer \cite{MEP:10}

\subsubsection{Thread Synchronization Modes}\label{SEC_threading}

Per-node sync, draw sync, async

%----------------------------------------------------------------------
\section{Applications}
%----------------------------------------------------------------------

\subsection{Livre}
\subsection{RTT Deltagen}

\subsection{RTNeuron}
\cite{HBBES:13}

\subsection{Raster}
\subsection{Omegalib}

%----------------------------------------------------------------------
\section{Experimental Results}
%----------------------------------------------------------------------
\label{SEC_results}

\subsection{Object Distribution}

Distribute large ply model to 1..16 nodes using TCP, IB and RSP

\subsection{Decomposition Modes}

Two graphs: Time to render a 4k, 256-step MSAA image of largest ply model and
Livre using 1..16 nodes with all modes (2D, 2D LB, DB, DPlex (over AA steps),
tiles, chunks, pixel, subpixel) (+DFR for Livre)

%----------------------------------------------------------------------
\section{Discussion and Conclusion}
%----------------------------------------------------------------------
\label{SEC_conclusions}


%----------------------------------------------------------------------
\appendices
% use section* for acknowledgment
\ifCLASSOPTIONcompsoc
  % The Computer Society usually uses the plural form
  \section*{Acknowledgments}
\else
  % regular IEEE prefers the singular form
  \section*{Acknowledgment}
\fi
%----------------------------------------------------------------------
We would like to thank and acknowledge the following institutions and projects
for providing the 3D geometry and volume test data sets: the Digital
Michelangelo Project, Stanford 3D Scanning Repository, Cyberware Inc.,
volvis.org and the Visual Human Project.  This work was partially supported by
the Swiss National Science Foundation Grant 200021-116329/1.

% trigger a \newpage just before the given reference
% number - used to balance the columns on the last page
% adjust value as needed - may need to be readjusted if
% the document is modified later
%\IEEEtriggeratref{8}
% The "triggered" command can be changed if desired:
%\IEEEtriggercmd{\enlargethispage{-5in}}

% biography section
%
% If you have an EPS/PDF photo (graphicx package needed) extra braces are
% needed around the contents of the optional argument to biography to prevent
% the LaTeX parser from getting confused when it sees the complicated
% \includegraphics command within an optional argument. (You could create
% your own custom macro containing the \includegraphics command to make things
% simpler here.)
%\begin{IEEEbiography}[{\includegraphics[width=1in,height=1.25in,clip,keepaspectratio]{mshell}}]{Michael Shell}
% or if you just want to reserve a space for a photo:

\begin{IEEEbiography}{Michael Shell}
Biography text here.
\end{IEEEbiography}

% if you will not have a photo at all:
\begin{IEEEbiographynophoto}{John Doe}
Biography text here.
\end{IEEEbiographynophoto}

% insert where needed to balance the two columns on the last page with
% biographies
%\newpage

\begin{IEEEbiographynophoto}{Jane Doe}
Biography text here.
\end{IEEEbiographynophoto}

\bibliographystyle{abbrv}
\bibliography{references}

% that's all folks
\end{document}
